\documentclass[a4paper,12pt]{article}

%\usepackage{amsmath,amssymb,multicol,tikz,enumitem}
\usepackage[margin=2cm]{geometry}
%\usetikzlibrary{calc}
\usepackage{amsmath}
\usepackage{amsthm}
\usepackage{thmtools}
\usepackage{hyperref}
\usepackage{enumerate}

\pagestyle{empty}

\newcommand\Q{\mathbf{Q}}
\newcommand\R{\mathbf{R}}
\newcommand\Z{\mathbf{Z}}

\usepackage{array}
\newcolumntype{P}[1]{>{\centering\arraybackslash}p{#1}}

\newcommand\indd{${}$\hspace{20pt}}

\declaretheoremstyle[headfont=\normalfont\bfseries,notefont=\mdseries\bfseries,bodyfont = \normalfont,headpunct={:}]{normalhead}
\declaretheorem[name={Uzdevums}, style=normalhead,numberwithin=section]{problem}

\setcounter{section}{1}

\setlength\parindent{0pt}

\begin{document}

\begin{center}
\parbox{3.5cm}{\flushleft\bf Skaitļu teorija\linebreak NMS juniori} \hfill {\bf\LARGE Mājasdarbs \#1} \hfill \parbox{3.5cm}{\flushright\bf 2020./2021.m.g.} \\[2pt]
\rm\small 2020.gada 10.oktobris
\end{center}

%\hrule\vspace{2pt}\hrule
\hrule

\vspace{20pt}
{\bf Iesniegšanas termiņš:} 2020.g.\ 31.oktobris\\
{\bf Kam iesūtīt:} {\tt kalvis.apsitis}, domēns {\tt gmail.com}


\vspace{20pt}
\begin{problem}
Dota kopa $S = \{ 105,106,\ldots,210 \}$. Noteikt mazāko
naturālo $n$ vērtību, ka, izvēloties jebkuru $n$ skaitļu
apakškopu $T$ no kopas $S$, tajā būs vismaz divi skaitļi, kuri nav
savstarpēji pirmskaitļi.
\end{problem}

\vspace{20pt}
\begin{problem} 
Visiem naturāliem skaitļiem $m > n$ pierādīt, ka
$$\mbox{MKD}(m,n) + \mbox{MKD}(m+1,n+1) > \frac{2mn}{\sqrt{m-n}}.$$
(Ar $\operatorname{MKD}(a,b)$ apzīmē naturālu skaitļu $a$ un $b$ {\em mazāko 
kopīgo dalāmo} \textendash{} mazāko skaitli, kas dalās gan ar $a$, gan ar $b$.)
\end{problem}

\vspace{20pt}
\begin{problem}
Vai eksistē bezgalīga
stingri augoša naturālu skaitļu virkne $a_1 < a_2 < a_3 <\ldots$,
ka jebkuram fiksētam naturālam skaitlim $a$ virknē 
$$a_1+a < a_2+a < a_3 + a < \ldots$$
ir ne vairāk kā galīgs skaits pirmskaitļu?
\end{problem}

\vspace{20pt}
\begin{problem}
Pierādīt, ka virkne $1,11,111,\ldots$ satur bezgalīgu apakšvirkni,
kuras katri divi locekļi ir savstarpēji pirmskaitļi.
\end{problem}



\vspace{20pt}
\begin{problem}
Pierādīt vai apgāzt sekojošus apgalvojumus:
\begin{enumerate}
\item Jebkuram $k \geq 2$, un jebkuriem $k$ pēc kārtas sekojošiem naturāliem 
skaitļiem atradīsies skaitlis, kurš nedalās ne ar vienu pirmskaitli, kas mazāks par $k$.   
\item Jebkuram $k \geq 2$, un jebkurai $k$ pēc kārtas sekojošu naturālu skaitļu virknei atradīsies skaitlis, kas ir savstarpējs pirmskaitlis ar visiem citiem virknes locekļiem. 
\end{enumerate}
\end{problem}

\newpage


{\bf (1) Galīgas ģeometriskas progresijas summa.} Ģeometrisko progresiju ar pirmo locekli $b_0$ 
un kvocientu $q$ definē ar formulu: $b_k = b_0 \cdot q^k$ (naturāliem $k>0$), t.i.\ katru nākamo locekli 
iegūst, reizinot iepriekšējo ar $q$. Šādas progresijas pirmo $k$ locekļu summu izsaka ar formulu:
$$\sum\limits_{j=0}^{k-1} b_0 q^j = b_0 + b_0 \cdot q + b_0 \cdot q^2 + \ldots + q^{k-1} = b_0 \frac{q^k-1}{q-1}.$$

{\bf (2) Pakāpju starpības dalīšana reizinātājos.} 
Ja naturāls skaitlis $k$ dalās ar naturālu skaitli $m$, bet $a,b$ ir jebkādi (reāli vai naturāli) mainīgie, 
tad algebrisku izteiksmi $a^k - b^k$ var sadalīt reizinātājos tā, ka viens no reizinātājiem ir $a^m - b^m$.\\
Piemērs: Ja $k = 6$ un $m=2$, tad var lietot kubu starpības formulu:
$$a^6 - b^6 = (a^2)^3 - (b^2)^3 = \left(a^2 - b^2\right)\left(a^4 + a^2b^2 + b^4\right).$$

{\bf (3) Nevienādība par aritmētisko un ģeometrisko vidējo.}
Ja $a,b$ ir jebkuri pozitīvi (reāli) skaitļi, tad ir spēkā nevienādība:
${\displaystyle \frac{a+b}{2} \leq \sqrt{a \cdot b}}$.

{\bf (4) Divi blakusesoši skaitļi ir savstarpēji pirmskaitļi.}
Ja $n$ ir jebkurš naturāls skaitlis, tad lielākais kopīgais dalītājs $\operatorname{LKD}(n,n+1) = 1$.

{\bf (5) LKD un MKD īpašības.}\\
{\bf (a)} Ja $\operatorname{LKD}(a,b)=\operatorname(a,c)=1$, tad 
$\operatorname{LKD}(a,bc)=1$.\\
{\bf (b)} Ja $\operatorname{LKD}(a,b)=1$ un $c\,\mid\,a$, tad 
$\operatorname{LKD}(b,c)=1$.\\
{\bf (c)} Ja $\operatorname{LKD}(a,b)=1$, tad $\operatorname{LKD}(ac,b)
= \operatorname{LKD}(c,b)$.\\
{\bf (d)} Ja $\operatorname{LKD}(a,b)=1$ un $c\,\mid\,(a+b)$, tad
$\operatorname{LKD}(a,c) = \operatorname{LKD}(b,c)=1$.\\
{\bf (e)} Ja $\operatorname{LKD}(a,b)=1$, $d\,\mid\,ac$ un 
$d\,\mid\,bc$, tad $d\,\mid\,c$.\\
{\bf (f)} Ja $\operatorname{LKD}(a,b)=1$, 
tad $\operatorname{LKD}\left(a^2,b^2\right)=1$.\\
{\bf (g)} $\operatorname{LKD}(a,b)\cdot\operatorname{MKD}(a,b)=a \cdot b$.\\
{\bf (h)} Distributivitātes likumi abām operācijām vienai pret otru:\\
\mbox{}\hspace{20pt}$\operatorname{MKD}(a,\operatorname{LKD}(b,c)) = \operatorname{LKD}(\operatorname{MKD}(a,b),\operatorname{MKD}(a,c))$,\\
\mbox{}\hspace{20pt}$\operatorname{LKD}(a,\operatorname{MKD}(b,c)) = \operatorname{MKD}(\operatorname{LKD}(a,b),\operatorname{LKD}(a,c))$.\\
{\bf (i)} Ja $p$ ir pirmskaitlis, tad $\text{LKD}(p,m)$ ir 
$p$ vai $1$.\\
{\bf (j)} Ja $\text{LKD}(m,n) = d$, tad $m/d$ un $n/d$ ir
savstarpēji pirmskaitļi.\\
{\bf (k)} Ja $m/d^{\ast}$ un $n/d^{\ast}$ abi ir veseli 
un savstarpēji pirmskaitļi, tad 
$\text{LKD}(m,n) = d^{\ast}$.\\
{\bf (l)} $\text{LKD}(m,n) = \text{LKD}(m-n,n)$. LKD
nemainās, ja no viena skaitļa atņem otru skaitli 
(arī divkāršotu, trīskāršotu utt. otru skaitli).



{\bf (6) Ķīniešu atlikumu teorēma.}\\
Ja $a,b,c$ ir naturāli skaitļi, kuri ir pa pāriem savstarpēji pirmskaitļi ($\operatorname{LKD}(a,b)=1$, 
$\operatorname{LKD}(a,c)=1$, $\operatorname{LKD}(b,c)=1$), tad jebkuriem trim skaitļiem 
$$a_1 \in [0,a-1],\;\;b_1 \in [0,b-1],\;\;c_1 \in [0,c-1],$$
atradīsies tāds $M$, kurš dod šos atlikumus $a_1,b_1,c_1$, dalot attiecīgi ar $a,b,c$. 
(Līdzīgu rezultātu var vispārināt; nav noteikti jāņem trīs savstarpēji pirmskaitļi $a,b,c$, 
bet var būt divi, četri, pieci vai vairāk.)\\
{\em Piemērs:} Atrast skaitli, kas dod atlikumu $9$, dalot ar $11$, atlikumu $3$, dalot ar $13$ un 
atlikumu $1$, dalot ar $17$. ($11,13,17$ ir savstarpēji pirmskaitļi, tātad mūsu atlikumu vēlmes nav pretrunīgas.)
Visi skaitļi formā $11k+9$ dod atlikumu $9$, dalot ar $11$:\\
$9,20,31,42,53,64,75,86,97,108,119,130,141,\ldots$
Starp tiem ir skaitlis $42$, kas dod ar atlikumu $3$, dalot ar $13$.\\
Vispārīgā formā visi skaitļi $11 \cdot 13 \cdot k + 42$ dos vēlamos atlikumus, 
gan dalot ar $11$, gan, dalot ar $13$. 
Visbeidzot, ievērojam, ka ievietojot $k=16$, iegūstam 
$11 \cdot 13 \cdot k + 42 = 2330$, kas dod 
sākumā izraudzītos atlikumus $9,3,1$, dalot attiecīgi ar $11,13,17$. 





\end{document}