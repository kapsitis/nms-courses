\documentclass[a4paper,12pt]{article}

%\usepackage{amsmath,amssymb,multicol,tikz,enumitem}
\usepackage[margin=2cm]{geometry}
%\usetikzlibrary{calc}
\usepackage{amsmath}
\usepackage{amsthm}
\usepackage{thmtools}
\usepackage{hyperref}
\usepackage{enumerate}
\usepackage{xcolor}

\pagestyle{empty}

\newcommand\Q{\mathbf{Q}}
\newcommand\R{\mathbf{R}}
\newcommand\Z{\mathbf{Z}}

\usepackage{array}
\newcolumntype{P}[1]{>{\centering\arraybackslash}p{#1}}

\newcommand\indd{${}$\hspace{20pt}}

\declaretheoremstyle[headfont=\normalfont\bfseries,notefont=\mdseries\bfseries,bodyfont = \normalfont,headpunct={:}]{normalhead}
\declaretheorem[name={Uzdevums}, style=normalhead,numberwithin=section]{problem}

\setcounter{section}{3}

\setlength\parindent{0pt}

\begin{document}

\begin{center}
\parbox{3.5cm}{\flushleft\bf Skaitļu teorija\linebreak NMS juniori} \hfill {\bf\LARGE Mājasdarbs \#3} \hfill \parbox{3.5cm}{\flushright\bf 2020./2021.m.g.} \\[2pt]
\rm\small 2021.gada 31.janvāris
\end{center}

%\hrule\vspace{2pt}\hrule
\hrule

\vspace{10pt}
{\bf Iesniegšanas termiņš:} 2021.g.\ 20.februāris\\
{\bf Kam iesūtīt:} {\tt kalvis.apsitis}, domēns {\tt gmail.com}

\vspace{10pt}
\begin{problem}
% Yuhong; p55, ex3
Pierādīt, ka jebkuram naturālam skaitlim $n$, ir $n$ pēc kārtas sekojoši 
naturāli skaitļi, ka jebkuram no tiem ir dalītājs, kas ir pilns kvadrāts, kas lielāks par $1$. 
\end{problem}

\vspace{10pt}
\begin{problem}
% Yuhong; p57, ex5
Dotajam naturālam skaitlim $n$, ar $f(n)$ apzīmējam mazāko naturālo skaitli, ka 
${\displaystyle \sum\limits_{k=1}^{f(n)} k}$ dalās ar $n$. 
Pierādīt, ka $f(n) = 2n-1$ tad un tikai tad, ja $n$ ir skaitļa $2$ pakāpe.
\end{problem}

\vspace{10pt}
\begin{problem}
% Yuhong; p58, ex6
Ar $n$ un $k$ apzīmējam veselus skaitļus, ka $n>0$ un skaitlis $k(n-1)$ ir pāra skaitlis. 
Pierādīt, ka eksistē skaitļi $x$ un $y$, ka $\text{LKD}(x,n) = \text{LKD}(y,n) = 1$ un 
$x + y \equiv k\;\pmod{n}$. 
\end{problem}

\vspace{10pt}
\begin{problem}
Plakne sadalīta kvadrātiņos kā rūtiņu papīra lapa; rūtiņas garums ir $1$ un vienā no rūtiņu 
virsotnēm ir koordinātu sākumpunkts.
Sauksim rūtiņu virsotni $X$ šajā plaknē par {\em redzamu} no koordinātu sākumpunkta 
$O(0;0)$, ja nogrieznis $OX$ nesatur citas rūtiņu virsotnes ar abām veselām koordinātēm, 
izņemot $O$ un $X$. Pierādīt, ka jebkuram naturālam $n$ eksistē kvadrāts ar $n \times n$ rūtiņu virsotnēm
(kur kvadrāta malas ir paralēlas koordinātu asīm), 
ka neviena no šīm $n^2$ rūtiņu virsotnēm nav redzama no koordinātu sākumpunkta.
\end{problem}

\vspace{10pt}
\begin{problem}
Ar $m, n$ apzīmēti naturāli skaitļi, kas apmierina šādu īpašību:
$$ \text{LKD}(11k-1,m) = \text{LKD}(11k-1,n)\;\;\text{ir spēkā visiem naturāliem skaitļiem $k$.}$$
Pierādīt, ka $m = 11^rn$ kādam veselam skaitlim $r$. 
\end{problem}






\end{document}









