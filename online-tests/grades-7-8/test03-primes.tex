\documentclass[11pt]{article}
\usepackage{ucs}
\usepackage[utf8x]{inputenc}
\usepackage{changepage}
\usepackage{graphicx}
\usepackage{amsmath}
\usepackage{gensymb}
\usepackage{amssymb}
\usepackage{enumerate}
\usepackage{tabularx}
\usepackage{lipsum}

\oddsidemargin 0.0in
\evensidemargin 0.0in
\textwidth 6.27in
\headheight 1.0in
\topmargin 0.0in
\headheight 0.0in
\headsep 0.0in
%\textheight 9.69in
\textheight 9.00in

\setlength\parindent{0pt}

\newenvironment{myenv}{\begin{adjustwidth}{0.4in}{0.4in}}{\end{adjustwidth}}
\renewcommand{\abstractname}{Anotācija}
\renewcommand\refname{Atsauces}

\newenvironment{uzdevums}[1][\unskip]{%
\vspace{3mm}
\noindent
\textbf{#1:}
\noindent}
{}

\newcommand{\subf}[2]{%
  {\small\begin{tabular}[t]{@{}c@{}}
  #1\\#2
  \end{tabular}}%
}



\newcounter{alphnum}
\newenvironment{alphlist}{\begin{list}{(\Alph{alphnum})}{\usecounter{alphnum}\setlength{\leftmargin}{2.5em}} \rm}{\end{list}}


\makeatletter
\let\saved@bibitem\@bibitem
\makeatother

\usepackage{bibentry}
%\usepackage{hyperref}


\begin{document}

\begin{center}
{\LARGE \bf Skaitļu teorija}
\end{center}

\begin{center}
{\large Jautājumu lapa \#003 (Pirmskaitļi)}
\end{center}


\begin{uzdevums}[AOPS.INT.2.1]
Četriem veseliem skaitļiem $\{2,4,10,x\}$ piemīt īpašība, ka ikvienu trīs locekļu summa, ja tai pieskaita vēl 1, ir pirmskaitlis. 
Kāda ir mazākā iespējamā $x$ vērtība, ja zināms, ka $x > 10$?  
\end{uzdevums}





\begin{uzdevums}[AOPS.INT.2.2]
Skaitlis $m$ ir trīsciparu naturāls skaitlis un tas ir trīs atšķirīgu pirmskaitļu reizinājums - $x$, $y$ un $10x+y$, 
kur $x$ un $y$ ir mazāki par 10. Kāda ir lielākā iespējamā $m$ vērtība?
\end{uzdevums}



\begin{uzdevums}[AOPS.INT.2C.1]
Atrast atlikumu, ja sešu mazāko pirmskaitļu summu dala ar septīto pirmskaitli. 
\end{uzdevums}

\begin{uzdevums}[AOPS.INT.2C.2]
Skaitlis 13 ir pirmskaitlis. Ja tā ciparus pieraksta pretējā secībā, arī iegūst pirmskaitli, 31. Kāds ir lielākais pirmskaitļu 
pāris, kas iegūstami viens no otra, apmainot ciparus, ja viņu summa ir 110? 
\end{uzdevums}

\begin{uzdevums}[AOPS.INT.2C.3]
Kāds ir mazākais pirmskaitlis, ar kuru dalās $5^{23} + 7^{17}$?
\end{uzdevums}

\begin{uzdevums}[AOPS.INT.2C.4]
25 vienādu monētu kolekcija salikta trīs kaudzītēs tā, ka ikvienā kaudzītē monētu skaits ir cits pirmskaitlis. Kāds monētu 
skaits ir iespējams lielākajā no kaudzītēm? 
\end{uzdevums}

\begin{uzdevums}[AOPS.INT.2C.5]
Ir pavisam 25 pirmskaitļi no 1 līdz 100. Kādai daļai no šiem pirmskaitļiem ciparu summa dalās ar 9?
\end{uzdevums}

\begin{uzdevums}[AOPS.INT.2C.6]
Vai 9409 pirmskaitlis?
\end{uzdevums}

\begin{uzdevums}[AOPS.INT.2C.7]
Kāds ir lielākais divciparu pirmskaitlis, kura katrs cipars arī ir pirmskaitlis?
\end{uzdevums}

\begin{uzdevums}[AOPS.INT.2C.8]
Kāds gads ir pirmais gadskaitlis 21.gadsimtā, kurš ir pirmskaitlis? 
\end{uzdevums}


\end{document}