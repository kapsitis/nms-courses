\documentclass[11pt]{article}
\usepackage{ucs}
\usepackage[utf8x]{inputenc}
\usepackage{changepage}
\usepackage{graphicx}
\usepackage{amsmath}
\usepackage{gensymb}
\usepackage{amssymb}
\usepackage{enumerate}
\usepackage{tabularx}
\usepackage{lipsum}

\oddsidemargin 0.0in
\evensidemargin 0.0in
\textwidth 6.27in
\headheight 1.0in
\topmargin 0.0in
\headheight 0.0in
\headsep 0.0in
%\textheight 9.69in
\textheight 9.00in

\setlength\parindent{0pt}

\newenvironment{myenv}{\begin{adjustwidth}{0.4in}{0.4in}}{\end{adjustwidth}}
\renewcommand{\abstractname}{Anotācija}
\renewcommand\refname{Atsauces}

\newenvironment{uzdevums}[1][\unskip]{%
\vspace{3mm}
\noindent
\textbf{#1:}
\noindent}
{}

\newcommand{\subf}[2]{%
  {\small\begin{tabular}[t]{@{}c@{}}
  #1\\#2
  \end{tabular}}%
}



\newcounter{alphnum}
\newenvironment{alphlist}{\begin{list}{(\Alph{alphnum})}{\usecounter{alphnum}\setlength{\leftmargin}{2.5em}} \rm}{\end{list}}


\makeatletter
\let\saved@bibitem\@bibitem
\makeatother

\usepackage{bibentry}
%\usepackage{hyperref}


\begin{document}

\begin{center}
{\LARGE \bf Skaitļu teorija}
\end{center}

\begin{center}
{\large Jautājumu lapa \#005 (Dalīšana pirmreizinātājos)}
\end{center}


\begin{uzdevums}[AOPS.INT.4.1]
Kurš ir lielākais pirmskaitlis $6! + 8!$. (Ja $n$ ir naturāls skaitlis, tad $n!$ ir reizinājums 
$1\cdot 2\cdot 3\cdot \cdots \cdot (n-1)\cdot n$.)
\end{uzdevums}


\begin{uzdevums}[AOPS.INT.4.2]
Cik daudzus daļskaitļus $\frac{n}{99}$, kur $0<n<99$, nevar saīsināt? 
\end{uzdevums}


\begin{uzdevums}[AOPS.INT.4.3]
Kurš ir mazākais pilnais kvadrāts, kas dalās ar 3 dažādiem pirmskaitļiem? (Skaitli sauc par {\em pilnu kvadrātu}, ja tas ir $n^2$ kādam naturālam $n$.)
\end{uzdevums}


\begin{uzdevums}[AOPS.INT.4.4]
Skolas orķestris noskaidroja, ka viņi var sarindoties 6, 7 vai 8 vienādās rindās tā, lai neviens nepaliek pāri. 
Kāds ir mazākais skolēnu skaits šajā orķestrī?
\end{uzdevums}


\begin{uzdevums}[AOPS.INT.4.5]
Kāds ir mazākais veselais skaitlis, kurš dalās ar 7, bet dod atlikumu 1, ja to dala ar jebkuru skaitli no 2 līdz 6? 
\end{uzdevums}


\begin{uzdevums}[AOPS.INT.4.6]
Divi velosipēdisti sāk braukt no kopīgas starta līnijas 12:15. Vienam velosipēdistam vajag $12$ minūtes, lai apbrauktu apli, 
kamēr otrs velosipēdists apbrauc apli katras $16$ minūtes. Pieņemot, ka viņu ātrumi ir nemainīgi, kad būs nākamais brīdis, kad 
viņi abi reizē šķērsos šo starta līniju? Atbildi izsakiet formā $hh:mm$. 
\end{uzdevums}

\begin{uzdevums}[AOPS.INT.4.7]
Atrast mazāko četrciparu skaitli, kas dalās ar katru no četriem mazākajiem pirmskaitļiem. 
\end{uzdevums}

\begin{uzdevums}[AOPS.INT.4.8]
Divus skaitļus sauc par ``relatīviem pirmskaitļiem'', ja to lielākais kopīgais dalītājs ir $1$. Cik daudzi veselie skaitļi no 1 līdz 28 ir 
relatīvi pirmskaitļi ar 28? 
\end{uzdevums}


\begin{uzdevums}[AOPS.INT.4-6.16]
Sadalīt pirmreizinātājos katru no skaitļiem $14^2$, $15^2$, $16^2$, $17^2$, un $18^2$.
\end{uzdevums}


\begin{uzdevums}[AOPS.INT.4-6.18]
Atrast piecus mazākos skaitļa 8 daudzkārtņus, kuri ir pilni kvadrāti. 
\end{uzdevums}



\end{document}