\documentclass[11pt]{article}
\usepackage{ucs}
\usepackage[utf8x]{inputenc}
\usepackage{changepage}
\usepackage{graphicx}
\usepackage{amsmath}
\usepackage{gensymb}
\usepackage{amssymb}
\usepackage{enumerate}
\usepackage{tabularx}
\usepackage{lipsum}

\oddsidemargin 0.0in
\evensidemargin 0.0in
\textwidth 6.27in
\headheight 1.0in
\topmargin 0.0in
\headheight 0.0in
\headsep 0.0in
%\textheight 9.69in
\textheight 9.00in

\setlength\parindent{0pt}

\newenvironment{myenv}{\begin{adjustwidth}{0.4in}{0.4in}}{\end{adjustwidth}}
\renewcommand{\abstractname}{Anotācija}
\renewcommand\refname{Atsauces}

\newenvironment{uzdevums}[1][\unskip]{%
\vspace{3mm}
\noindent
\textbf{#1:}
\noindent}
{}

\newcommand{\subf}[2]{%
  {\small\begin{tabular}[t]{@{}c@{}}
  #1\\#2
  \end{tabular}}%
}



\newcounter{alphnum}
\newenvironment{alphlist}{\begin{list}{(\Alph{alphnum})}{\usecounter{alphnum}\setlength{\leftmargin}{2.5em}} \rm}{\end{list}}


\makeatletter
\let\saved@bibitem\@bibitem
\makeatother

\usepackage{bibentry}
%\usepackage{hyperref}


\begin{document}

\begin{center}
{\LARGE \bf Skaitļu teorija}
\end{center}

\begin{center}
{\large Jautājumu lapa \#001 (Veselie skaitļi, Dalāmība)}
\end{center}


\begin{uzdevums}[nt.div.01]
% AOPS.INT.1.1
Cik daudzi pozitīvi trīsciparu skaitļi dalās gan ar 11, gan ar 5? 
\end{uzdevums}


\begin{uzdevums}[nt.div.02]
% AOPS.INT.1.2
Kurš ir mazākais pozitīvais skaitļa 25 daudzkārtnis, kura ciparu reizinājums arī ir pozitīvs skaitļa 25 daudzkārtnis?
\end{uzdevums}


\begin{uzdevums}[nt.div.03]
% AOPS.INT.1.3
Kuri no apgalvojumiem ir patiesi?
\begin{enumerate}[A.]
\item 3 ir skaitļa 18 dalītājs.
\item 17 ir skaitļa 187 dalītājs, bet nav skaitļa 52 dalītājs.
\item 24 nav ne skaitļa 72, ne arī skaitļa 67 dalītājs.
\item 13 ir 26 dalītājs, bet nav 52 dalītājs. 
\item 8 ir skaitļa 160 dalītājs.
\end{enumerate}
{\em  Pierakstiet savu atbildi ar burtiem alfabētiskā secībā, atdalot burtus ar komantiem. Piemēram, ja uzskatāt, ka visi pieci apgalvojumi ir patiesi, rakstiet "A,B,C,D,E" (bez pēdiņām).}
\end{uzdevums}


\begin{uzdevums}[nt.div.04]
% AOPS.INT.1.4
Visi skaitļa 175 pozitīvie dalītāji, izņemot 1, ir izrakstīti pa apli tā, ka jebkuriem diviem skaitļiem 
blakus uz apļa ir kopīgs reizinātājs, 
kas lielāks par 1. Kāda ir summa abiem skaitļiem, kuri uzrakstīti blakus skaitlim 7?
\end{uzdevums}



\begin{uzdevums}[nt.div.05]
% AOPS.INT.1.5
Orķestrī spēlē 72 skolēni, kuri visi soļos sporta spēles pārtraukumā. Viņiem jāsoļo rindās -- ar vienādu skolēnu skaitu katrā rindā. 
Vienā rindā jābūt 5--20 skolēniem. Cik dažādus rindu garumus orķestra dalībnieki var izveidot?
\end{uzdevums}


\begin{uzdevums}[nt.div.06]
% AOPS.INT.1.6
Pieņemsim, ka $a$ un $b$ ir veseli pozitīvi skaitļi, kur skaitlim $a$ ir 3 dalītāji, bet skaitlim $b$ ir $a$ dažādi dalītāji. 
Ja $b$ dalās ar $a$, tad kāda var būt skaitļa $b$ mazākā iespējamā vērtība? 
\end{uzdevums}


\begin{uzdevums}[nt.div.07]
% AOPS.INT.1.7
Kāds ir lielākais veselais skaitlis, ar kuru dalās jebkuru trīs pēc kārtas sekojošu naturālu skaitļu reizinājums? 
\end{uzdevums}


\begin{uzdevums}[nt.div.08]
% AOPS.INT.1.8
Naturāli skaitļi $A,$ $B,$ $A-B$ un $A+B$ visi ir pirmskaitļi. Šo četru pirmskaitļu summa ir\\
{\bf (A)} pāra skaitlis;\hspace{1ex} {\bf (B)} dalās ar $3$;\hspace{1ex} {\bf (C)} dalās ar $5$;\hspace{1ex} 
{\bf (D)} dalās ar $7$;\hspace{1ex} {\bf (E)} ir pirmskaitlis\\
{\em Pierakstiet savu atbildi ar burtiem A, B, C, D, un E. Ja der vairāki burti, atdaliet tos ar komatiem.}
\end{uzdevums}

\begin{uzdevums}[nt.div.09]
% AOPS.INT.1.9
Ar $m$ un $n$ apzīmējam attiecīgi lielāko un mazāko skaitļa 7 daudzkārtni starp visiem trīsciparu skaitļiem. 
Kāda ir $m + n$ vērtība?
\end{uzdevums}


\begin{uzdevums}[nt.div.10]
% AOPS.INT.1.10
Programmas sākumā 105 orķestra dalībnieki sastājās "Taisnstūrī $A$" (visās rindās vienāds skaits dalībnieku). 
Pēc tam viņi pārkārtojās "Taisnstūrī $B$", kam rindu ir par 6 vairāk, 
bet katrā no rindām ir par diviem dalībniekiem mazāk. 
Cik rindu ir "Taisnstūrī $A$"? 
\end{uzdevums}


\end{document}