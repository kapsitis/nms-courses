\documentclass[11pt]{article}
\usepackage{ucs}
\usepackage[utf8x]{inputenc}
\usepackage{changepage}
\usepackage{graphicx}
\usepackage{amsmath}
\usepackage{gensymb}
\usepackage{amssymb}
\usepackage{enumerate}
\usepackage{tabularx}
\usepackage{lipsum}

\oddsidemargin 0.0in
\evensidemargin 0.0in
\textwidth 6.27in
\headheight 1.0in
\topmargin 0.0in
\headheight 0.0in
\headsep 0.0in
%\textheight 9.69in
\textheight 9.00in

\setlength\parindent{0pt}

\newenvironment{myenv}{\begin{adjustwidth}{0.4in}{0.4in}}{\end{adjustwidth}}
\renewcommand{\abstractname}{Anotācija}
\renewcommand\refname{Atsauces}

\newenvironment{uzdevums}[1][\unskip]{%
\vspace{3mm}
\noindent
\textbf{#1:}
\noindent}
{}

\newcommand{\subf}[2]{%
  {\small\begin{tabular}[t]{@{}c@{}}
  #1\\#2
  \end{tabular}}%
}



\newcounter{alphnum}
\newenvironment{alphlist}{\begin{list}{(\Alph{alphnum})}{\usecounter{alphnum}\setlength{\leftmargin}{2.5em}} \rm}{\end{list}}


\makeatletter
\let\saved@bibitem\@bibitem
\makeatother

\usepackage{bibentry}
%\usepackage{hyperref}


\begin{document}

\begin{center}
{\LARGE \bf Skaitļu teorija}
\end{center}

\begin{center}
{\large Jautājumu lapa \#007 ($\operatorname{gcd}(\cdots)$, $\operatorname{lcm}(\cdots)$}
\end{center}


\begin{uzdevums}[1]
Dženija novieto 42 brūnas Lieldienu olas vairākos oranžos grozos un 
$28$ zaļas Lieldienu olas vairākos zilos grozos. Katrā grozā ir vienāds olu skaits, un katrā grozā ir 
vismaz $4$ olas. 
\begin{enumerate}[(a)]
\item Cik olas Dženija ielika katrā no groziem?
\item Cik bija oranžo grozu?
\item Cik bija zilo grozu?
\end{enumerate}
\end{uzdevums}


\begin{uzdevums}[2]
Kādu nedēļu Doktors Patriks sadalīja $40$ studentu auditoriju vienādās grupās pa $n$ studentiem katrā grupā
($n > 1$). Nākamajā nedēļā 10 studenti uz nodarbību neieradās, bet Doktors Patriks joprojām varēja 
sadalīt atlikušos studentus grupās pa $n$ studentiem. Atrast visas iespējamās $n$ vērtības. 
\end{uzdevums}

\begin{uzdevums}[3]
Sadalīt skaitļus $6$, $8$, $10$, $15$, $21$ un $25$ trīs pāros tā, lai katrā pārī 
būtu {\em relatīvi pirmskaitļi} - t.i.\ abiem skaitļiem nebūtu kopīga dalītāja, kas lielāks par $1$. 
\end{uzdevums}

\begin{uzdevums}[4]
Jīnga un Katerīna iegāja veikalā, lai pirktu zīmuļus. Jīnga nopirka 40 zīmuļus un Katerīna nopirka 24 zīmuļus. 
Ja katrā zīmuļu paciņā ir vienāds zīmuļu skaits, kāds var būt lielākais zīmuļu skaits paciņā? 
(Var izmantot Eiklīda algoritmu - t.i. ja Jīngas un Katrīnas zīmuļu skaits dalās ar $n$, 
tad arī Jīngas un Katerīnas nopirkto zīmuļu skaita starpība dalās ar $n$.) 
\end{uzdevums}

\begin{uzdevums}[5]
Atrast visus skaitļu $6$ un $15$ kopīgos daudzkārtņus starp $1$ un $100$. 
\end{uzdevums}

\begin{uzdevums}[6]
Kāds ir mazākais cilvēku skaits, kurus var sadalīt gan $15$ vienādās grupās, gan arī $48$ vienādās grupās? 
\end{uzdevums}

\begin{uzdevums}[7]
Atrast lielāko divciparu skaitli, kas dod atlikumu $2$ dalot ar $7$. 
Atrast lielāko trīsciparu skaitli, kas dod atlikumu $4$ dalot ar $11$. 
\end{uzdevums}

\begin{uzdevums}[8]
Cik ir skaitļu no $1$ līdz $100$, kas dod atlikumu $1$, dalot ar $6$? 
Cik ir skaitļu no $1$ līdz $100$, kas dod atlikumu $5$, dalot ar $6$? 
\end{uzdevums}

\begin{uzdevums}[9]
Trīsciparu skaitlis $1\ast\ast$ sākas ar ciparu $1$ un dod atlikumu $5$, dalot ar $8$. 
Atrast, kādi cipari var būt zvaigznīšu vietā. 
\end{uzdevums}

\begin{uzdevums}[10]
Uz rūtiņu papīra uzzīmēti divi taisnstūri ar vienādu platumu, kam malas ir vesels skaits rūtiņu. 
Šo taisnstūru laukumi ir $1086$ un $828$. Kāds var būt šo taisnstūru kopīgais platums? 
\end{uzdevums}


\end{document}