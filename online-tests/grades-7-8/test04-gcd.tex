\documentclass[11pt]{article}
\usepackage{ucs}
\usepackage[utf8x]{inputenc}
\usepackage{changepage}
\usepackage{graphicx}
\usepackage{amsmath}
\usepackage{gensymb}
\usepackage{amssymb}
\usepackage{enumerate}
\usepackage{tabularx}
\usepackage{lipsum}

\oddsidemargin 0.0in
\evensidemargin 0.0in
\textwidth 6.27in
\headheight 1.0in
\topmargin 0.0in
\headheight 0.0in
\headsep 0.0in
%\textheight 9.69in
\textheight 9.00in

\setlength\parindent{0pt}

\newenvironment{myenv}{\begin{adjustwidth}{0.4in}{0.4in}}{\end{adjustwidth}}
\renewcommand{\abstractname}{Anotācija}
\renewcommand\refname{Atsauces}

\newenvironment{uzdevums}[1][\unskip]{%
\vspace{3mm}
\noindent
\textbf{#1:}
\noindent}
{}

\newcommand{\subf}[2]{%
  {\small\begin{tabular}[t]{@{}c@{}}
  #1\\#2
  \end{tabular}}%
}



\newcounter{alphnum}
\newenvironment{alphlist}{\begin{list}{(\Alph{alphnum})}{\usecounter{alphnum}\setlength{\leftmargin}{2.5em}} \rm}{\end{list}}


\makeatletter
\let\saved@bibitem\@bibitem
\makeatother

\usepackage{bibentry}
%\usepackage{hyperref}


\begin{document}

\begin{center}
{\LARGE \bf Skaitļu teorija}
\end{center}

\begin{center}
{\large Jautājumu lapa \#004 (Kopīgie dalītāji un dalāmie)}
\end{center}


\begin{uzdevums}[AOPS.INT.3R.24D]
Uzrakstīt visus naturālos skaitļus, kuri ir dalītāji gan skaitlim 36, gan skaitlim 84. 
(T.i.\ viņi ir 36 un 84 {\em kopīgie dalītāji}.)
\end{uzdevums}

\begin{uzdevums}[AOPS.INT.3R.25F]
Uzrakstīt sešus mazākos skaitļus, kuri dalās gan ar 18, gan ar 42. (T.i.\ šie seši skaitļi 
būs 18 un 42 {\em kopīgie dalāmie} jeb {\em kopīgie daudzkārtņi}.)
\end{uzdevums}

\begin{uzdevums}[AOPS.INT.3R.3]
Kāds ir skaitļu 12, 18 un 30 mazākais kopīgais dalāmais?
\end{uzdevums}

\begin{uzdevums}[AOPS.INT.3R.28]
Atrast skaitļu  $6432$ un $132$ lielāko kopīgo dalītāju. 
(Ja to uzreiz grūti ieraudzīt, 
varat atskaitīt no lielākā skaitļa mazāko, kas pareizināts, teiksim, ar 48. \\
$6432 - 132\times 48 = 96$. Un pēc tam meklēt lielāko kopīgo dalītāju 
skaitļiem $96$ un $132$. (Tas būs lielākais kopīgais dalītājs arī 
skaitļiem $6432$ un $132$.)
\end{uzdevums}

\begin{uzdevums}[AOPS.INT.3R.29]
Cik daudzi no 100 mazākajiem naturālajiem skaitļiem (t.i.\ skaitļi no $1$ līdz $100$)
dalās ar $7$?
\end{uzdevums}


\begin{uzdevums}[AOPS.INT.3.1]
Lūcija piedzima 2004.g. 1.decembrī, trešdienā. Šī trešdiena bija viņas dzīves pirmā diena. Viņas vecāki 
sarīkoja svinības viņas dzīves 1000.dienā. Kurā nedēļas dienā notika šīs svinības. 
\end{uzdevums}

\begin{uzdevums}[AOPS.INT.3.3]
Džakss nopirka tik daudz koku, lai pietiktu iestādīšanai astoņās vienādās rindās. 
Viens koks nomira un to nevarēja iestādīt, bet Džaksam tagad bija tik daudz koku, lai 
tos iestādītu deviņās vienādās rindās. Pēc tam vienu koku nozaga, bet Džaksam bija tik daudz 
koku, lai iestādītu desmit vienādas rindas. Ja zināms, ka Džakss nopirka mazāko koku daudzumu, 
kas izpilda šos trīs nosacījumus, cik daudz koku viņam bija pašā sākumā?
\end{uzdevums}

\begin{uzdevums}[AOPS.INT.3.4]
Ir zināmas algebras formulas
\[ \begin{array}{rcl}
a^5 - b^5 & = & (a-b) \left(a^4 + a^3b + a^2b^2 + ab^3 + b^4 \right), \\ 
a^6 - b^6 & = & (a-b) \left(a^5 + a^4b + a^3b^2 + a^2b^3 + ab^4 + b^5 \right). \\ 
\end{array} \]
Kāds ir lielākais kopīgais dalītājs skaitļiem $2^{30} - 1$ un $2^{25} - 1$? 
\end{uzdevums}

\begin{uzdevums}[AOPS.INT.3.5]
Naturāls skaitlis ir par 3 lielāks nekā skaitļa 4 daudzkārtnis, un vienlaikus - par 4 lielāks nekā skaitļa 5 daudzkārtnis. 
Kāds vismazākais skaitlis izpilda šīs abas īpašības?
\end{uzdevums}

\begin{uzdevums}[AOPS.INT.3.6]
Ik pēc 5 mēnešiem Halam jānomaina baterijas savā kalkulatorā. Viņš pirmo reizi tās nomainīja maijā. Kurā mēnesī tās tiks nomainītas 25.reizi?
\end{uzdevums}



\end{document}