\documentclass[11pt]{article}
\usepackage{ucs}
\usepackage[utf8x]{inputenc}
\usepackage{changepage}
\usepackage{graphicx}
\usepackage{amsmath}
\usepackage{gensymb}
\usepackage{amssymb}
\usepackage{enumerate}
\usepackage{tabularx}
\usepackage{lipsum}

\oddsidemargin 0.0in
\evensidemargin 0.0in
\textwidth 6.27in
\headheight 1.0in
\topmargin 0.0in
\headheight 0.0in
\headsep 0.0in
%\textheight 9.69in
\textheight 9.00in

\setlength\parindent{0pt}

\newenvironment{myenv}{\begin{adjustwidth}{0.4in}{0.4in}}{\end{adjustwidth}}
\renewcommand{\abstractname}{Anotācija}
\renewcommand\refname{Atsauces}

\newenvironment{uzdevums}[1][\unskip]{%
\vspace{3mm}
\noindent
\textbf{#1:}
\noindent}
{}

\newcommand{\subf}[2]{%
  {\small\begin{tabular}[t]{@{}c@{}}
  #1\\#2
  \end{tabular}}%
}



\newcounter{alphnum}
\newenvironment{alphlist}{\begin{list}{(\Alph{alphnum})}{\usecounter{alphnum}\setlength{\leftmargin}{2.5em}} \rm}{\end{list}}


\makeatletter
\let\saved@bibitem\@bibitem
\makeatother

\usepackage{bibentry}
%\usepackage{hyperref}


\begin{document}

\begin{center}
{\LARGE \bf Skaitļu teorija}
\end{center}

\begin{center}
{\large Jautājumu lapa \#006 (Dalīšana pirmreizinātājos)}
\end{center}


\begin{uzdevums}[1]
Sadalīt pirmreizinātājos skaitļus 99, 999 un 99999. T.i. izteikt tos kā pirmskaitļu pakāpju
reizinājumus: $p_1^{a_1} \cdot p_2^{a_2} \cdot \ldots \cdot p_n^{a_n}$, kur
$p_1 < p_2 < \ldots < p_n$ ir pirmskaitļi augošā secībā.
\end{uzdevums}


\begin{uzdevums}[2]
Atrast mazākos kopīgos dalāmos (jeb kopsaucējus): $\operatorname{lcm}(6,8,14)$ un 
$\operatorname{lcm}(9,12,16)$. (Šeit ar $\operatorname{lcm}(\ldots)$ apzīmē 
mazāko kopīgo dalāmo jeb {\em least common multiple}.)
\end{uzdevums}

\begin{uzdevums}[3]
Kurš ir mazākais četrciparu skaitlis, kurš dalās ar 
$2$, $3$, $4$, $5$, $6$ un $7$?
\end{uzdevums}

\begin{uzdevums}[4]
Zināms, ka $108 = 2^2 \cdot 3^3$ un $360 = 2^3 \cdot 3^2 \cdot 5^1$. 
Atrast 108 un 360 mazāko kopīgo dalāmo un lielāko kopīgo dalītāju. 
Izteikt tos kā pirmskaitļu pakāpju reizinājumus. 
\end{uzdevums}

\begin{uzdevums}[5]
Atrast $\gcd(42, 700)$, $\frac{42}{\gcd(42, 700)}$ un $\frac{700}{\gcd(42, 700)}$. 
(Šeit ar $\operatorname{gcd}(\ldots)$ apzīmē 
lielāko kopīgo dalītāju jeb {\em greatest common divisor}.)
\end{uzdevums}

\begin{uzdevums}[6]
Atrast piecus mazākos pilnos kvadrātus, kas dalās ar $6$.
\end{uzdevums}

\begin{uzdevums}[7]
Kamerons uzrakstīja mazāko skaitļa 20 daudzkārtni, kas ir pilns kvadrāts (t.i. $n^2$), 
mazāko skaitļa 20 daudzkārtni, kas ir pilns kubs (t.i. $n^3$) un vēl arī 
visus skaitļa 20 daudzkārtņus, kuri atrodas starp šiem abiem skaitļiem. 
Cik skaitļus Kamerons uzrakstīja? 
\end{uzdevums}

\begin{uzdevums}[8]
Pieņemsim, ka $n$ ir naturāls, skaitlis, kuram $\gcd(n, 70) = 10$ un $\mathrm{lcm}(n, 70) = 210$. 
Atrast $n$. (Šeit $\gcd(\ldots)$ un $\operatorname{lcm}(\ldots)$
apzīmē attiecīgi lielāko kopīgo dalītāju un mazāko kopīgo dalāmo.)
\end{uzdevums}

\begin{uzdevums}[9]
Sk. $n$ no iepriekšējā uzdevuma. Kurš skaitlis ir lielāks: $n \cdot 70$ vai arī
$\gcd(n, 70) \cdot \operatorname{lcm}(n,70)$.  
\end{uzdevums}

\begin{uzdevums}[10]
Saīsināt šo daļu:
$\dfrac{770}{100100}$.
\end{uzdevums}


\end{document}