\documentclass[11pt]{article}
\usepackage{ucs}
\usepackage[utf8x]{inputenc}
\usepackage{changepage}
\usepackage{graphicx}
\usepackage{amsmath}
\usepackage{gensymb}
\usepackage{amssymb}
\usepackage{enumerate}
\usepackage{tabularx}
\usepackage{lipsum}

\oddsidemargin 0.0in
\evensidemargin 0.0in
\textwidth 6.27in
\headheight 1.0in
\topmargin 0.0in
\headheight 0.0in
\headsep 0.0in
%\textheight 9.69in
\textheight 9.00in

\setlength\parindent{0pt}

\newenvironment{myenv}{\begin{adjustwidth}{0.4in}{0.4in}}{\end{adjustwidth}}
\renewcommand{\abstractname}{Anotācija}
\renewcommand\refname{Atsauces}

\newenvironment{uzdevums}[1][\unskip]{%
\vspace{3mm}
\noindent
\textbf{#1:}
\noindent}
{}

\newcommand{\subf}[2]{%
  {\small\begin{tabular}[t]{@{}c@{}}
  #1\\#2
  \end{tabular}}%
}



\newcounter{alphnum}
\newenvironment{alphlist}{\begin{list}{(\Alph{alphnum})}{\usecounter{alphnum}\setlength{\leftmargin}{2.5em}} \rm}{\end{list}}


\makeatletter
\let\saved@bibitem\@bibitem
\makeatother

\usepackage{bibentry}
%\usepackage{hyperref}


\begin{document}

\begin{center}
{\LARGE \bf Skaitļu teorija}
\end{center}

\begin{center}
{\large Jautājumu lapa \#002 (Veselie skaitļi, Dalāmība)}
\end{center}

\begin{uzdevums}[nt.div.11]
% AOPS.INT.1C.1
Laila, Sandra, Ilze un Mārtiņš kopā mācījās algebru. Pirmajā kontroldarbā Laila dabūja 94, Sandra dabūja 91, Ilze dabūja 95, un 
Mārtiņa rezultāts bija starp 81 un 87 (galapunktus ieskaitot). Ja zināms, ka viņu rezultātu aritmētiskais 
vidējais ir vesels skaitlis, kāds bija Mārtiņa rezultāts pirmajā kontroldarbā?
\end{uzdevums}

\begin{uzdevums}[nt.div.12]
% AOPS.INT.1C.2
Atrast visus pilnos kubus starp 1000 un 2000. 
\end{uzdevums}

\begin{uzdevums}[nt.div.13]
% AOPS.INT.1C.3
Kāds ir lielākais veselais skaitlis, kura kubs ir mazāks par 10000? 
\end{uzdevums}

\begin{uzdevums}[nt.div.14]
% AOPS.INT.1C.4
Cik daudzi veseli skaitļi no 1 līdz 100 ir pilnas pakāpes (lielākas par pirmo)?
\end{uzdevums}

\begin{uzdevums}[nt.div.15]
% AOPS.INT.1C.5
Cik veseli skaitļi no 1 līdz 9 ir piecciparu skaitļa 24516 dalītāji? 
\end{uzdevums}

\begin{uzdevums}[nt.div.16]
% AOPS.INT.1C.6
Vai 11111 dalās ar 41? 
Kuri cipari periodiski atkārtojas decimāldaļskaitlī $1/41$? 
\end{uzdevums}

\begin{uzdevums}[nt.div.17]
% AOPS.INT.1C.7
Is 111111 divisible by 37? Kuri cipari periodiski atkārtojas decimāldaļskaitlī $1/37$? 
\end{uzdevums}

\begin{uzdevums}[nt.div.18]
% AOPS.INT.1C.8
Atrast mazāko naturālo skaitli, kurš nav 5040 dalītājs. 
\end{uzdevums}

\begin{uzdevums}[nt.div.19]
% AOPS.INT.1C.9
Riņķa diametrs ir vesels skaitlis. Riņķa laukums ir skaitlis starp 100 un 120. 
Cik garš ir riņķa diametrs?
({\em Riņķa laukumu aprēķina pēc formulas $S = \pi r^2$, 
kur $\pi \approx 3.14$ un $r$ ir riņķa rādiuss.})
\end{uzdevums}

\begin{uzdevums}[nt.div.20]
% AOPS.INT.1C.10
Izteikt 112 kā četru pilnu kvadrātu summu. 
\end{uzdevums}




\end{document}