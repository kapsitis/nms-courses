\documentclass[11pt]{article}
\usepackage{ucs}
\usepackage[utf8x]{inputenc}
\usepackage{changepage}
\usepackage{graphicx}
\usepackage{amsmath}
\usepackage{gensymb}
\usepackage{amssymb}
\usepackage{enumerate}
\usepackage{tabularx}
\usepackage{lipsum}
\usepackage{amsthm}
\usepackage{thmtools}


\usepackage{fontspec} % loaded by polyglossia, but included here for transparency 
\usepackage{polyglossia}

\usepackage{xeCJK}
\setCJKmainfont{SimSun}
\setmainlanguage{russian} 
\setotherlanguage{english}

\newfontfamily\cyrillicfont[Script=Cyrillic]{Times New Roman}
\newfontfamily\cyrillicfontsf[Script=Cyrillic]{Arial}
\newfontfamily\cyrillicfonttt[Script=Cyrillic]{Courier New}

\oddsidemargin 0.0in
\evensidemargin 0.0in
\textwidth 6.27in
\headheight 1.0in
\topmargin 0.0in
\headheight 0.0in
\headsep 0.0in
%\textheight 9.69in
\textheight 9.00in
 
\setlength\parindent{0pt}

\newenvironment{myenv}{\begin{adjustwidth}{0.4in}{0.4in}}{\end{adjustwidth}}
\renewcommand{\abstractname}{Anotācija}
\renewcommand\refname{Atsauces}

%\newenvironment{uzdevums}[1][\unskip]{%
%\vspace{3mm}
%\noindent
%\textbf{#1:}
%\noindent}
%{}


% http://tex.stackexchange.com/questions/196961/thmtools-declaration-for-theorem-and-proof
\declaretheoremstyle[headfont=\normalfont\bfseries,notefont=\mdseries\bfseries,bodyfont = \normalfont,headpunct={:}]{normalhead}
\declaretheorem[name={Uzdevums}, style=normalhead,numberwithin=section]{problem}

\def\changemargin#1#2{\list{}{\rightmargin#2\leftmargin#1}\item[]}
\let\endchangemargin=\endlist 


\newcommand{\subf}[2]{%
  {\small\begin{tabular}[t]{@{}c@{}}
  #1\\#2
  \end{tabular}}%
}



\newcounter{alphnum}
\newenvironment{alphlist}{\begin{list}{(\Alph{alphnum})}{\usecounter{alphnum}\setlength{\leftmargin}{2.5em}} \rm}{\end{list}}

\newenvironment{zhtext}{\fontfamily{MS PGothic}\selectfont}{\par}


\makeatletter
\let\saved@bibitem\@bibitem
\makeatother

\usepackage{bibentry}
%\usepackage{hyperref}

\newenvironment{tulkojums}[1][\unskip]{%
\begin{changemargin}{8mm}{8mm}
\fontsize{9}{11}
\selectfont
\textbf{#1:}
}
{ 
\fontsize{12}{14}
\selectfont
\end{changemargin}
}

\setcounter{section}{0}

\begin{document}

\section{Annas tests \#1}


%% "IMO1994.P3" "IMO1994.P4" "IMO1995.P6" 
%% "IMO1999.P4" "IMO2002.P3" "IMO2003.P2" 
%% "IMO2007.P5" "IMO2011.P5" "IMO2012.P4" 
%% "IMO2013.P1" "IMO2015.P2" "IMO2016.P4"

\begin{problem}
\[
\left\{
\begin{array}{lcl}
a + b & = & 80 \\
a - b & = & 30 \\
\end{array}
\right.
\]
{\bf Atrisinājums:} $a = \rule{10ex}{.4pt}$; $b = \rule{10ex}{.4pt}$.
\end{problem}

\begin{problem}
\[
\left\{
\begin{array}{lcl}
(a+b)(a - b) & = & 1400 \\
a + b & = & 100 \\
\end{array}
\right.
\]
{\bf Atrisinājums:} $a - b = \rule{10ex}{.4pt}$; $a = \rule{10ex}{.4pt}$; $b = \rule{10ex}{.4pt}$.
\end{problem}

\begin{problem}
\[
\left\{
\begin{array}{lcl}
a^2 - b^2 & = & 19 \\
a + b & = & 19 \\
\end{array}
\right.
\]
{\bf Atrisinājums:} $a - b = \rule{10ex}{.4pt}$; $a = \rule{10ex}{.4pt}$; $b = \rule{10ex}{.4pt}$.
\end{problem}

\begin{problem}
\[
\left\{
\begin{array}{lcl}
a + b & = & 60 \\
a^2 + b^2 & = & 2000 \\
\end{array}
\right.
\]
{\bf Atrisinājums:} $(a + b)^2 = \rule{10ex}{.4pt}$; $2ab = \rule{10ex}{.4pt}$; $ab = \rule{10ex}{.4pt}$\\[5mm]
$a^2 - 2ab + b^2 = \rule{10ex}{.4pt}$; $a - b = \rule{10ex}{.4pt}$; $a = \rule{10ex}{.4pt}$;
$b = \rule{10ex}{.4pt}$.
\end{problem}


\begin{problem}
Taisnstūrim laukums ir $1600\operatorname{cm}^2$, bet perimetrs $200\operatorname{cm}$.\\[5mm]
{\bf Atrisinājums:} Viena mala ir $\rule{10ex}{.4pt}$; otra mala ir $\rule{10ex}{.4pt}$
\end{problem}

\begin{problem}
\[
\left\{
\begin{array}{lcl}
a + b & = & 5 \\
ab & = & 6 \\
\end{array}
\right.
\]
{\bf Atrisinājums:} $(a + b)^2 = \rule{10ex}{.4pt}$; $2ab = \rule{10ex}{.4pt}$; \\[5mm]
$a^2 - 2ab + b^2 = \rule{10ex}{.4pt}$; $a - b = \rule{10ex}{.4pt}$; $a = \rule{10ex}{.4pt}$;
$b = \rule{10ex}{.4pt}$.
\end{problem}

\begin{problem}
Izteikt zīmējumā redzamās figūras laukumu (to veido divi kvadrāti un divi taisnleņķa trijstūri) ar $a$ un $b$.
\begin{center}
\includegraphics[width=1.5in]{shape1.png}
\end{center}
{\bf Atrisinājums:} Laukums ir $\rule{10ex}{.4pt}$;
\end{problem}

\begin{problem}
Ja $a = b$, tad izteiksmes $a^2 + b^2$ un $2ab$ ir vienādas. Bet kura izteiksme ir lielāka, ja $a \neq b$?  \\[5mm]
{\bf Atrisinājums:}  $a^2 + b^2$ ir (lielāka/mazāka) par $2ab$, jo $\rule{10ex}{.4pt}$
\end{problem}
 


\end{document}


