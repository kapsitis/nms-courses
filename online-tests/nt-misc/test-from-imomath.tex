\documentclass[11pt]{article}
\usepackage{ucs}
\usepackage[utf8x]{inputenc}
\usepackage{changepage}
\usepackage{graphicx}
\usepackage{amsmath}
\usepackage{gensymb}
\usepackage{amssymb}
\usepackage{enumerate}
\usepackage{tabularx}
\usepackage{lipsum}
\usepackage{amsthm}
\usepackage{thmtools}


\usepackage{fontspec} % loaded by polyglossia, but included here for transparency 
\usepackage{polyglossia}

\usepackage{xeCJK}
\setCJKmainfont{SimSun}
\setmainlanguage{russian} 
\setotherlanguage{english}

\newfontfamily\cyrillicfont[Script=Cyrillic]{Times New Roman}
\newfontfamily\cyrillicfontsf[Script=Cyrillic]{Arial}
\newfontfamily\cyrillicfonttt[Script=Cyrillic]{Courier New}

\oddsidemargin 0.0in
\evensidemargin 0.0in
\textwidth 6.27in
\headheight 1.0in
\topmargin 0.0in
\headheight 0.0in
\headsep 0.0in
%\textheight 9.69in
\textheight 9.00in
 
\setlength\parindent{0pt}

\newenvironment{myenv}{\begin{adjustwidth}{0.4in}{0.4in}}{\end{adjustwidth}}
\renewcommand{\abstractname}{Anotācija}
\renewcommand\refname{Atsauces}

%\newenvironment{uzdevums}[1][\unskip]{%
%\vspace{3mm}
%\noindent
%\textbf{#1:}
%\noindent}
%{}


% http://tex.stackexchange.com/questions/196961/thmtools-declaration-for-theorem-and-proof
\declaretheoremstyle[headfont=\normalfont\bfseries,notefont=\mdseries\bfseries,bodyfont = \normalfont,headpunct={:}]{normalhead}
\declaretheorem[name={Uzdevums}, style=normalhead,numberwithin=section]{problem}

\def\changemargin#1#2{\list{}{\rightmargin#2\leftmargin#1}\item[]}
\let\endchangemargin=\endlist 


\newcommand{\subf}[2]{%
  {\small\begin{tabular}[t]{@{}c@{}}
  #1\\#2
  \end{tabular}}%
}



\newcounter{alphnum}
\newenvironment{alphlist}{\begin{list}{(\Alph{alphnum})}{\usecounter{alphnum}\setlength{\leftmargin}{2.5em}} \rm}{\end{list}}

\newenvironment{zhtext}{\fontfamily{MS PGothic}\selectfont}{\par}


\makeatletter
\let\saved@bibitem\@bibitem
\makeatother

\usepackage{bibentry}
%\usepackage{hyperref}

\newenvironment{tulkojums}[1][\unskip]{%
\begin{changemargin}{8mm}{8mm}
\fontsize{9}{11}
\selectfont
\textbf{#1:}
}
{ 
\fontsize{12}{14}
\selectfont
\end{changemargin}
}

\setcounter{section}{4}

\begin{document}

\section{Tests no imomath.com}



\begin{problem}[ImoMathCom.NT.1]
Pieņemsim, ka $m$ un $n$ ir naturāli skaitļi ($m>1$) tādi ka 
funkcijas $f(x) = \arcsin(\log_m(nx))$ ir slēgts intervāls garumā $\frac{1}{2013}$. 
Ar $S$ apzīmēsim $m+n$ mazāko iespējamo vērtību. 
Atrast atlikumu $S$ dalot ar $1000$. 
\end{problem}


\begin{problem}[ImoMathCom.NT.2]
Atrast mazāko naturālo $n$ tādu, ka, nodzēšot šī skaitļa pirmo ciparu (t.i. ciparu kreisajā pusē), 
iegūstam jaunu naturālu skaitli, kas vienāds ar $n/29$. 
\end{problem}


\begin{problem}[ImoMathCom.NT.3]
Ar $a,b,c,d$ apzīmēti reāli pozitīvi skaitļi, kam $a^2 + b^2 - c^2 - d^2 = 0$ un 
$a^2 - b^2 - c^2 + d^2 = \frac{56}{53}(bc+ad)$. Ar $M$ apzīmēsim 
lielāko iespējamo vērtību izteiksmei $\frac{ab+cd}{bc+ad}$. 
Ja $M$ izteikts kā nesaīsināma daļa $\frac{m}{n}$, atrast $100m+n$. 
\end{problem}

\begin{problem}[ImoMathCom.NT.4]
Ar $\tau(n)$ apzīmējam naturāla skaitļa $n$ dalītāju skaitu, ieskaitot $1$ un $n$. 
Apzīmējam $S(n) = \tau(1) + \tau(2) + \ldots + \tau(n)$. 
Ar $a$ apzīmējam naturālo skaitļu ($n \leq 2008$) skaitu, kam $S(n)$ ir nepāru, un ar $b$ apzīmējam 
naturālo skaitļu ($n \leq 2008$) skaitu, kam $S(n)$ ir pāru. Atrast $|a-b|$. 
\end{problem}

\begin{problem}[ImoMathCom.NT.5]
Cik ir tādu veselu skaitļu pāru $(x,y)$, kam $x^2−y^2=2400^2$? 
\end{problem}

\end{document}


