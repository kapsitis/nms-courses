\documentclass[a4paper,12pt]{article}

\usepackage{amsmath,amssymb,multicol,tikz,enumitem}
\usepackage[margin=2cm]{geometry}
%\usetikzlibrary{calc}
\usepackage{amsmath}
\usepackage{amsthm}
\usepackage{thmtools}
\usepackage{hyperref}
\usepackage{enumerate}
\usepackage{xcolor}
\usepackage{fancyvrb}

\pagestyle{empty}

\newcommand\Q{\mathbf{Q}}
\newcommand\R{\mathbf{R}}
\newcommand\Z{\mathbf{Z}}

%\newcommand\answer[1]{}
%\newcommand\ans[1]{}
\newcommand\answer[1]{\\[5pt]{\color{blue}{#1}}\hfill{\color{blue}}\\[-5pt]}
\newcommand\ans[1]{{\color{blue}{#1}}}

\usepackage{array}
\newcolumntype{P}[1]{>{\centering\arraybackslash}p{#1}}

\newcommand\indd{${}$\hspace{20pt}}

\declaretheoremstyle[headfont=\normalfont\bfseries,notefont=\mdseries\bfseries,bodyfont = \normalfont,headpunct={:}]{normalhead}
\declaretheorem[name={Uzdevums}, style=normalhead,numberwithin=section]{problem}

\setcounter{section}{05}

\setlength\parindent{0pt}

\renewcommand{\figurename}{Attēls}

\begin{document}

\begin{center}
\parbox{3.5cm}{\flushleft\bf Dažāda skaitļu teorija} \hfill {\bf\LARGE Sacensības \#2021.05} \hfill \parbox{3.5cm}{\flushright\bf 2021-04-29} \\[2pt]
{\rm\footnotesize Par šo LU NMS atbalstīto pasākumu\\ atbild {\tt kalvis.apsitis@gmail.com}.}
\end{center}

%\hrule\vspace{2pt}\hrule
\hrule





%%%%%%%%%%%%%%%%%%%%%%%%%%%
%%% Divisibility#1 %%%%%%%%
%%%%%%%%%%%%%%%%%%%%%%%%%%%
\vspace{10pt}
\begin{problem}
%2020.1.12
Ar $n$ apzīmējam mazāko naturālo skaitli, kuram $149^n - 2^n$ dalās 
ar $3^3 \cdot 5^5 \cdot 7^7$. Atrast cik skaitlim $n$ ir veselu pozitīvu dalītāju. 
\end{problem}




%%%%%%%%%%%%%%%%%%%%%%%%%%%
%%% Divisibility#2 %%%%%%%%
%%%%%%%%%%%%%%%%%%%%%%%%%%%
\vspace{10pt}
\begin{problem}
%2005.2.4
Atrast, cik ir tādu naturālu skaitļu, kuri dala vismaz vienu no skaitļiem
$10^{10}$, $15^{7}$, $18^{11}$.
\end{problem}


%%%%%%%%%%%%%%%%%%%%%%%%%%
%%% Congruences#1 %%%%%%%%
%%%%%%%%%%%%%%%%%%%%%%%%%%
\vspace{10pt}
\begin{problem}
%2021.2.13
Atrast mazāko naturālo $n$, kuram $2^n + 5^n - n$ 
dalās ar $1000$. 
\answer{

{\bf Atbilde.} $\mathtt{797}$.

Šajā uzdevumā nevar tieši izmantot Mazo Fermā vai Eilera teorēmu, 
jo gan $2$, gan $5$ ir kopīgi dalītāji ar $1000$. 
Tāpēc analizējam citādi: Sākam ar novērojumu, ka $n$ ir nepāru skaitlis, 
jo citādi $2^n$ un $n$ būtu pāra skaitļi, bet $5^n$ ir nepāra
(un summa nedalītos ar $1000$). 

Tā kā $n$ ir nepāra (un mazas vērtības $n=1,2$ uzdevuma 
nosacījumus neapmierina), tad $5^n \equiv 125$.
Tāpēc jārisina kongruenču vienādojums: 
\[ 2^{2k+1} + 125 - (2k+1) \equiv 0 \pmod{1000}. \]
Šeit apzīmēts $n = 2k+1$. 

Vispirms risinām šo kongruenci pēc $10$ moduļa, tad pēc $100$ moduļa, tad 
pēc $1000$ moduļa. 
Pakāpeniski iegūstam, ka $k \equiv 8 \pmod{10}$, tad 
$k \equiv 98 \pmod{100}$ un visbeidzot $k \equiv 398 \pmod{1000}$. 

Pēdējā kongruence nozīmē, ka $n = 2k+1$ ir kongruents ar $797$ pēc $1000$ moduļa
(t.i.\ $797$ ir mazākais skaitlis).
}
\end{problem}



%%%%%%%%%%%%%%%%%%%%%%%%%%
%%% Congruences#2 %%%%%%%%
%%%%%%%%%%%%%%%%%%%%%%%%%%
\vspace{10pt}
\begin{problem}
%2001.2.10
Cik daudzi naturāli skaitļi, kas dalās ar $1001$ var būt izteikti formā 
$10^j - 10^i$, kur $i,j$ ir veseli skaitļi, $0 \leq i < j \leq 99$? 
\end{problem}


\end{document}









