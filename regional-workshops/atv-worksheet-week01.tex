\documentclass[a4paper,12pt]{article}

%\usepackage{amsmath,amssymb,multicol,tikz,enumitem}
\usepackage[margin=2cm]{geometry}
%\usetikzlibrary{calc}
\usepackage{amsmath}
\usepackage{amsthm}
\usepackage{thmtools}
\usepackage{hyperref}
\usepackage{enumerate}
\usepackage{xcolor}
\usepackage{fancyvrb}

\pagestyle{empty}

\newcommand\Q{\mathbf{Q}}
\newcommand\R{\mathbf{R}}
\newcommand\Z{\mathbf{Z}}

\usepackage{array}
\newcolumntype{P}[1]{>{\centering\arraybackslash}p{#1}}

\newcommand\indd{${}$\hspace{20pt}}

\declaretheoremstyle[headfont=\normalfont\bfseries,notefont=\mdseries\bfseries,bodyfont = \normalfont,headpunct={:}]{normalhead}
\declaretheorem[name={Uzdevums}, style=normalhead,numberwithin=section]{problem}

\setcounter{section}{100}
\setcounter{problem}{0}


\setlength\parindent{0pt}

\begin{document}

\begin{center}
\parbox{3.5cm}{\flushleft\bf Skaitļu teorija \newline ATV} \hfill {\bf\LARGE Uzdevumu lapa \#1} \hfill \parbox{3.5cm}{\flushright\bf 2021-01-06} %\\[2pt]
%\rm\small 2020.gada 15.novembris
\end{center}

%\hrule\vspace{2pt}\hrule
\hrule

%\vspace{10pt}
%{\bf Iesniegšanas termiņš:} 2020.g.\ 2.janvāris\\
%{\bf Kam iesūtīt:} {\tt kalvis.apsitis}, domēns {\tt gmail.com}

%\vspace{5pt}
%{\em (Par racionāliem un iracionāliem skaitļiem, skaitļu pierakstu un periodiskām virknēm.)}


\vspace{10pt}
\begin{problem}
Ar $s(k)$ apzīmēsim naturāla skaitļa $k$ ciparu summu. 
Pierādīt, ka ir bezgalīgi daudz tādu naturālu skaitļu $n$, kas nedalās ar $10$ un 
kuriem ${\displaystyle s(n^2) < s(n) -5}$. 
\end{problem}


\vspace{10pt}
\begin{problem} 
Dots naturāls skaitlis $m$ un pirmskaitlis $p$, kas ir skaitļa $m^2 - 2$ dalītājs. 
Zināms, ka eksistē tāds naturāls skaitlis $a$, ka $a^2 + m - 2$ dalās ar $p$. 
Pierādīt, ka eksistē tāds naturāls skaitlis $b$, ka $b^2 - m - 2$ 
dalās ar $p$. 
\end{problem}

\vspace{10pt}
\begin{problem}
Atrodiet visus veselu skaitļu trijniekus $(a,b,c)$, kuriem 
$$(a-b)^3(a+b)^2 = c^2 + 2(a - b) + 1.$$
\end{problem}





\vspace{10pt}
\begin{problem}
Atrodiet visus naturālu skaitļu četriniekus $(x,y,z,t)$, kuri apmierina vienādojumu sistēmu
$$\left\{ 
\begin{array}{l}
xyz = t!\\
(x+1)(y+1)(z+1) = (t+1)!\\
\end{array} 
\right.$$
\end{problem}



\end{document}