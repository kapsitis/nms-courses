\documentclass[a4paper,12pt]{article}

%\usepackage{amsmath,amssymb,multicol,tikz,enumitem}
\usepackage[margin=2cm]{geometry}
%\usetikzlibrary{calc}
\usepackage{amsmath}
\usepackage{amsthm}
\usepackage{thmtools}
\usepackage{hyperref}
\usepackage{enumerate}
\usepackage{xcolor}
\usepackage{fancyvrb}

\pagestyle{empty}

\newcommand\Q{\mathbf{Q}}
\newcommand\R{\mathbf{R}}
\newcommand\Z{\mathbf{Z}}

\usepackage{array}
\newcolumntype{P}[1]{>{\centering\arraybackslash}p{#1}}

\newcommand\indd{${}$\hspace{20pt}}

\declaretheoremstyle[headfont=\normalfont\bfseries,notefont=\mdseries\bfseries,bodyfont = \normalfont,headpunct={:}]{normalhead}
\declaretheorem[name={Uzdevums}, style=normalhead,numberwithin=section]{problem}

\setcounter{section}{101}

\setlength\parindent{0pt}

\begin{document}


Matemātiski apgalvojumi par naturāliem skaitļiem parasti ir {\em vispārīgi} \textendash{}
tie attiecas uzreiz uz visu naturālo skaitļu kopu $\mathbf{N}$ (vai vismaz kādu bezgalīgu apakškopu: 
visiem pirmskaitļiem, visiem pāru skaitļiem, utml.) 
Pat apgalvojumi par galīgu skaitļu kopu (visiem trīsciparu skaitļiem) ir pietiekoši vispārīgi: 
tos nav praktiski iespējams atrisināt, aplūkojot visus gadījumus, ja neizmanto datoru. 
Vispārīgu apgalvojumu pierādīšanai jāizmanto metodes, kas piemērotas bezgalīgām kopām.

Matemātiskā indukcija balstās uz naturālo skaitļu {\em indukcijas aksiomu}, tā ir viena no t.s.\
Peano aksiomām, tāpēc pati indukcija ir princips, kas piemīt naturāliem skaitļiem (atbilstoši pašam 
šo skaitļu jēdzienam), šis princips nav pierādāms.
Pēc būtības indukcijas aksioma apgalvo, ka katru naturālo skaitli var iegūt, 
vienu vai vairākas reizes pieskaitot skaitlim 0 vieninieku. 
Citiem vārdiem sakot, naturālie skaitļi ir skaitļi, kas rodas skaitīšanas rezultātā: $1,2,3,\ldots$. 
Vēl citiem vārdiem \textendash{} indukcijas aksioma apgalvo, ka nav 
``nesasniedzamu'' skaitļu, līdz katram var nonākt, pieskaitot vieninieku. 

Matemātisko indukciju kā uzdevumu risināšanas metodi pieraksta šādi:
Pieņemsim, ka mums ir jāpierāda kāds vispārīgs apgalvojums $P(n)$, kurš kaut ko apgalvo par naturālu skaitli $n$.\\
{\bf 1.solis:} Vispirms pierāda apgalvojumu $P(1)$  ({\em indukcijas bāze});\\
{\bf 1.solis:} Pieņem, ka izpildās $P(k)$ ({\em induktīvais pieņēmums}), un tad pierāda, 
ka pie šī pieņēmuma izpildās arī apgalvojums $P(k+1)$ ({\em induktīvā pāreja}).

Ja šie abi punkti ir pierādīti, tad no matemātiskās indukcijas principa seko, 
ka apgalvojums $P(n)$ izpildās visiem naturāliem skaitļiem $n$.
Šo pierādījuma shēmu apzīmēsim šādi:\\ 
$1\,\parallel\,k \rightarrow k+1$, atdalot nepieciešamos
soļus ar vertikālām svītriņām.
Iespējamas arī citas indukcijas shēmas:
\begin{itemize}
\item
$0\,\parallel\,k \rightarrow k+1$ (iepriekšējās shēmas variants, ja jāsāk pierādīt jau no $P(0)$.\\
Līdzīgi var sākt arī no jekbkuras citas vietas; piemēram, var būt shēma $7\,\parallel\,k \rightarrow k+1$, 
bet tad atsevišķi jāpamato daži citi gadījumi $P(1),\ldots,P(6)$. 
\item
$1\,\parallel\,2\,\parallel\,\ldots\,\parallel\,a\,\parallel\,k \rightarrow k+a$; 
jāpārbauda apgalvojums pirmajiem $a$ skaitļiem, un jāpārbauda, ka no apgalvojuma $P(k)$ seko apgalvojums $P(k+a)$.
\item 
$1\,\parallel\,\;<k \rightarrow k+1$; jāpārbauda, ka izpildās apgalvojums $P(1)$, un, pieņemot, 
ka apgalvojums izpildās visiem naturāliem skaitļiem $n$, kas ir mazāki par $k$, 
jāpierāda, ka tas izpildās arī skaitlim $k$.
\end{itemize}

Pierādot jebkuru vispārīgu apgalvojumu par naturāliem skaitļiem formāli ir jāizmanto ma\-te\-mā\-tis\-kā indukcija, 
tāpēc liela daļa uzdevuma krājuma uzdevumi varētu būt ievietoti šajā nodaļā. 
Taču mēs šajā nodaļā esam apkopojuši tikai tos uzdevumus, 
kuros matemātiskās indukcijas princips ir būtiskākā atrisinājuma sastāvdaļa (un nav ``noslēpts'', izmantojot
kādu citu rezultātu, kas arī balstās uz indukciju). 
Pirmajās trīs apakšnodaļās aplūkojam dažādas indukcijas shēmas. 

Ceturtā apakšnodaļa pievēršas neparastam paņēmienam matemātiskajā indukcijā: 
ar indukciju vispirms pierāda pastiprinātu apgalvojumu (kas reizēm ir izdevīgi, jo 
ne tikai pierādāmais apgalvojums ir pastiprināts, bet arī induktīvais pieņēmums ir pastiprināts; 
izejot no tā, var vairāk pierādīt.)
Risinot šādus uzdevumus, vispirms formulēsim apgalvojumu, 
no kura seko uzdevuma apgalvojums, un pēc tam pierādīsim to ar indukciju.
Šādu metodi sauc par {\em induktīvās hipotēzes pastiprināšanu}, 
sk. \url{https://bit.ly/3nrKz2u}.



\clearpage
\begin{center}
\parbox{3.5cm}{\flushleft\bf Skaitļu teorija \newline ATV} \hfill {\bf\LARGE Uzdevumu lapa \#2} \hfill \parbox{3.5cm}{\flushright\bf 2021-01-13} %\\[2pt]
%\rm\small 2020.gada 15.novembris
\end{center}

%\hrule\vspace{2pt}\hrule
\hrule


\vspace{10pt}
\begin{problem}
%9.1. 
Pierādiet, ka visiem naturāliem skaitļiem $n$ skaitlis $3^{3n+3} - 26n - 27$ dalās ar $169$.
\end{problem}


\vspace{10pt}
\begin{problem}
%9.2. 
Atrodiet lielāko kopīgo dalītāju visiem skaitļiem $7^{n+2} + 8^{2n+1}$, 
ja $n$ ir vesels nenegatīvs skaitlis.
\end{problem}


\vspace{10pt}
\begin{problem}
%9.3. 
Pierādiet, ka visiem naturāliem skaitļiem $n$ skaitlis 
${\displaystyle 3\left( 1^5 + 2^5 + \cdots + n^5 \right)}$ dalās ar
${\displaystyle \left( 1^3 + 2^3 + \cdots + n^3 \right)}$.
\end{problem}


\vspace{10pt}
\begin{problem}
%9.4. 
Pierādiet, ka skaitlis $k^{2^n}-1$ dalās ar $2^{n+2}$, 
ja $k$ ir nepāra skaitlis, bet $n$ ir naturāls skaitlis.
\end{problem}



\vspace{10pt}
\begin{problem}
%9.5. 
Pierādiet, ka visiem veseliem nenegatīviem skaitļiem $n$ skaitlis $2^{3^n}+1$ dalās ar $3^{n+1}$.
\end{problem}



\vspace{10pt}
\begin{problem}
%9.6. 
Pierādiet, ka eksistē bezgalīgi daudzi tādi naturāli skaitļi $n$, 
ka skaitlis $2^n+1$ dalās ar $n$, un atrodiet visus pirmskaitļus $n$, 
kuriem izpildās šī īpašība.
\end{problem}




\vspace{10pt}
\begin{problem}
%9.7. 
Pierādiet, ka katram naturālam skaitlim $a>1$ eksistē bezgalīgi daudz tādu naturālu skaitļu $n$, 
kuriem skaitlis $a^n+1$ dalās ar $n$.
\end{problem}



\vspace{10pt}
\begin{problem}
%9.8. 
Pierādiet, ka katram naturālam skaitlim $n \geq 3$ eksistē naturāls skaitlis $k$, 
kurš ir uzrakstāms kā $n$ dažādu savu dalītāju summa.
\end{problem}




\vspace{10pt}
\begin{problem}
%9.9. 
Pierādiet, ka jebkuru naturālu skaitli, kurš nepārsniedz $n!$ 
var uzrakstīt kā skaitļa $n!$ dalītāju summu, kas satur ne vairāk kā $n$ saskaitāmos.
\end{problem}




\vspace{10pt}
\begin{problem}
%9.10. 
Pierādiet, ka katrai augošai aritmētiskai progresijai, kas sastāv no na\-tu\-rā\-liem skaitļiem, 
eksistē $100$ pēc kārtas sekojoši locekļi, kas visi ir salikti skaitļi.
\end{problem}



\vspace{10pt}
\begin{problem}
%9.11. 
Dota virkne: $a_1=5,\;a_{n+1}=a_n^2$, ja $n \geq 1$. 
Pierādiet, ka skaitļu $a_{1981}$ un $a_{1980}$ pēdējie $1980$ cipari ir vienādi.
\end{problem}




\vspace{10pt}
\begin{problem}
%9.12.
Pierādīt, ka eksistē tāds naturāls skaitlis $n$, ka $n^2+1$ dalās ar $5^{1979}$.
\end{problem}


\vspace{10pt}
\begin{problem}
%9.13. 
Apzīmēsim ar $d_n$  skaitļa $n$ lielāko nepāra dalītāju. Aprēķināt summu
$$d_1 + d_2 + \cdots + d_{2^{100}}.$$
\end{problem}


	 .
	 

\vspace{10pt}
\begin{problem}
%9.14. 
Pierādiet, ka no skaitļiem $1,2,3,\ldots,3^n$ var izvēlēties $2^n$ skaitļus tā, 
ka nekādi trīs no tiem neveido aritmētisko progresiju.
\end{problem}



\vspace{10pt}
\begin{problem}
%9.15. 
Nogrieznis ar garumu $3^n$ ir sadalīts trīs vienādās daļās. 
Pirmo un trešo no tām sauksim par apzīmētām daļām. 
Katrs no apzīmētajiem nogriežņiem no jauna tiek sadalīts trīs 
vienādās daļās, no kurām pirmo un trešo sauc par apzīmētām, u.t.t., 
līdz tam brīdim, kamēr iegūsim nogriežņus ar garumu $1$. 
Apzīmēto nogriežņu galapunktus sauksim par apzīmētiem punktiem. 
Pierādiet, ka jebkuram naturālam skaitlim $k \leq 3^n$ 
var atrast divus apzīmētus punktus, attālums starp kuriem ir vienāds ar $k$.
\end{problem}


\vspace{10pt}
\begin{problem}
%9.16. 
Dota naturālu skaitļu virkne $a_1,a_2,a_3,\ldots$, kurai izpildās īpašība
$$a_{k+1} \leq 1 + a_1 + a_2 + \cdots + a_{k-1}\;\;\mbox{visiem}\;\;k>1.$$
Pierādiet, ka visus naturālos skaitļus var izteikt kā dažu (varbūt arī viena) atšķirīgu šīs virknes locekļu summu.
\end{problem}


\vspace{10pt}
\begin{problem}
%9.17. 
Dota naturālu skaitļu virkne $\left(a_n \right)$, 
kurai visiem naturāliem skaitļiem $n$ un $k$ skaitlis $a_{n+k} - a_n$ dalās ar $a_k$. 
Apzīmēsim skaitli $a_1a_2\cdots{}a_n$ ar $b_n$. 
Pierādiet, ka visiem naturāliem skaitļiem $n$ un $k$ skaitlis $b_{n+k}$ dalās ar $b_nb_k$.
\end{problem}


\vspace{10pt}
\begin{problem}
%9.18. 
Pierādīt šādu apgalvojumu: ja $n$ ir naturāls skaitlis, tad $n^3 + (n+1)^3 + (n+2)^3$ dalās ar $9$.
\end{problem}


\vspace{10pt}
\begin{problem}
%9.19. 
Simtciparu skaitlis vienāds ar savu ciparu summu plus visu iespējamo savu ciparu pāru reizinājumu summu 
plus visu iespējamo savu ciparu trijnieku reizinājumu summu plus utt. plus visu savu ciparu reizinājumu. 
Atrast visus šādus skaitļus.
\end{problem}


\vspace{10pt}
\begin{problem}
%9.20. 
Reālu skaitļu virkni definē šādi: $x_1$ un $x_2$ izvēlas patvaļīgi, un
$$x_{n+2} = \frac{x_nx_{n+1}}{3x_n - 2x_{n+1}},\;\;n=1,2,3,\ldots .$$
Atrast visas iespējas, kā var izvēlēties $x_1$ un $x_2$, lai bezgalīgi daudzi virknes $\left( x_n \right)$ 
locekļi būtu naturāli skaitļi.
\end{problem}


\vspace{10pt}
\begin{problem}
%9.21. 
Aprēķināt izteiksmes
\[\cfrac{1}{2-\cfrac{1}{2-\cfrac{1}{\ddots\;\;\;\;\cfrac{ \begin{array}{l} \ddots \\ 1 \\ \end{array}  }{2-\cfrac{1}{2}}}}}.\]
vērtību, izsakot to kā nesaīsināmu daļu.
\end{problem}


\vspace{10pt}
\begin{problem}
%9.22. 
Pierādīt, ka katram naturālam $n$\\
{\bf (a)} $1^3 + 2^3 + \cdots + n^3$ dalās ar $1 + 2 + \cdots +n$,\\
{\bf (b)} $1^5 + 2^5 + \cdots + n^5$ dalās ar $1 + 2 + \cdots +n$.
\end{problem}





\end{document}