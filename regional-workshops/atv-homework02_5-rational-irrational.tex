\documentclass[a4paper,12pt]{article}

%\usepackage{amsmath,amssymb,multicol,tikz,enumitem}
\usepackage[margin=2cm]{geometry}
%\usetikzlibrary{calc}
\usepackage{amsmath}
\usepackage{amsthm}
\usepackage{thmtools}
\usepackage{hyperref}
\usepackage{enumerate}
\usepackage{xcolor}
\usepackage{fancyvrb}

\pagestyle{empty}

\newcommand\Q{\mathbf{Q}}
\newcommand\R{\mathbf{R}}
\newcommand\Z{\mathbf{Z}}

\usepackage{array}
\newcolumntype{P}[1]{>{\centering\arraybackslash}p{#1}}

\newcommand\indd{${}$\hspace{20pt}}

\declaretheoremstyle[headfont=\normalfont\bfseries,notefont=\mdseries\bfseries,bodyfont = \normalfont,headpunct={:}]{normalhead}
\declaretheorem[name={Uzdevums}, style=normalhead,numberwithin=section]{problem}

\setcounter{section}{25}

\setlength\parindent{0pt}

\begin{document}

\begin{center}
\parbox{3.5cm}{\flushleft\bf Skaitļu teorija\linebreak ATV} \hfill {\bf\LARGE Uzdevumu lapa \#0} \hfill \parbox{3.5cm}{\flushright\bf 2020./2021.m.g.} %\\[2pt]
%\rm\small 2020.gada 15.novembris
\end{center}

%\hrule\vspace{2pt}\hrule
\hrule

%\vspace{10pt}
%{\bf Iesniegšanas termiņš:} 2020.g.\ 2.janvāris\\
%{\bf Kam iesūtīt:} {\tt kalvis.apsitis}, domēns {\tt gmail.com}

\vspace{5pt}
{\em (Par racionāliem un iracionāliem skaitļiem, skaitļu pierakstu un periodiskām virknēm.)}


\vspace{10pt}
\begin{problem}
Plakne sadalīta vienības kvadrātiņos kā rūtiņu papīra lapa. 
\begin{enumerate}[(A)]
\item Vai var šajā plaknē uzzīmēt regulāru piecstūri, kura visas virsotnes atrastos rūtiņu virsotnēs?
\item Vai šajā plaknē var uzzīmēt regulāru trijstūri, kura visas virsotnes atrastos rūtiņu virsotnēs?
\end{enumerate}
{\em Piezīme.} Par {\em rūtiņu virsotnēm} saucam tos plaknes punktus, kuros sastopas četri vienības kvadrātiņi.
\end{problem}


\vspace{10pt}
\begin{problem} 
Rūtiņu plaknē var iezīmēt dažādu izmēru kvadrātus (gan taisnus, gan slīpus), kuru virsotnes var atrasties rūtiņu 
virsotnēs. Šajos piemēros mūs interesē kvadrāti ar noteitiem izmēriem. Pieņemam, ka vienas rūtiņas izmērs ir $1 \times 1$
(viena garuma vienība). 
\begin{enumerate}[(A)]
\item
Vai eksistē kvadrāts ar izmēriem $17 \times 17$, kura virsotnes atrodas rūtiņu virsotnēs, bet malas nav paralēlas rūtiņu malām?
\item 
Vai eksistē kvadrāts ar diagonāles garumu $\sqrt{26}$, kura virsotnes atrodas rūtiņu virsotnēs?
\item 
Vai eksistē kvadrāts ar diagonāles garumu $\sqrt{76}$, kura virsotnes atrodas rūtiņu virsotnēs?
\item 
Vai eksistē kvadrāts ar diagonāles garumu $\sqrt{126}$, kura virsotnes atrodas rūtiņu virsotnēs?
\end{enumerate}
\end{problem}

\vspace{10pt}
\begin{problem}
Bezgalīgu virkni no nullēm un vieniniekiem veido sekojoši: 
\begin{itemize} 
\item Vispirms uzraksta ciparu $0$. 
\item Katrā nākamajā solī apskata līdzšinējās virknes locekļus, visu nuļļu vietā ieraksta vieniniekus, bet visu vieninieku vietā nulles \textendash{}
un pieraksta galā esošajai virknei. 
\end{itemize}
Veicot šos soļus, iegūstam meklējamās virknes posmus (katrs nākamais posms divreiz garāks par iepriekšējo):
\begin{verbatim}
0
01
0110
01101001
0110100110010110
01101001100101101001011001101001
...
\end{verbatim}
Ja virknes locekļus numurē, sākot ar $0$-to, iegūstam, ka 
$t_0 = 0$, $t_1 = 1$, $t_2 = 1$, $t_3 = 0$, $t_4 = 1$, $t_5 = 0$, $t_6 = 0$, $t_7 = 1$, 
$t_8 = 1$, $t_9 = 0$, utt. Lai atrastu jebkuru virknes locekli, uzraksta pietiekoši 
daudzus posmus šajā konstrukcijā. 
\begin{enumerate}[(A)]
\item Vai virkne $t_n$ ir periodiska? (Ja ir, atrast tās priekšperiodu un periodu. Ja nav, pamatot, kāpēc nav.)
\item Aplūkojam jaunu virkni ${\displaystyle g_n = t_{3^n}}$, t.i. tos $t_n$ locekļus, kuru numuri ir skaitļa $3$ pakāpes. 
Jau atradām, ka 
$$\left\{ \begin{array}{l}
g_0 = t_{3^0} = t_1 = 1,\\
g_1 = t_{3^1} =  t_3 = 0,\\
g_2 = t_{3^2} = t_9 = 0. 
\end{array} \right.$$
Vai virkne $g_n$ ir periodiska? (Ja ir, atrast tās priekšperiodu un periodu. Ja nav, pamatot, kāpēc nav.)
\end{enumerate}
\end{problem}



\vspace{10pt}
\begin{problem}
Ilzītei sākumā bija $10$ eirocenti. Viņa nopelnīja vēl $7$ eiras, bet pēc tam $7$ eiras iztērēja. 
Lai noskaidrotu, cik naudas atlicis, viņa rēķināja ar Python:
\begin{Verbatim}[frame=single,numbers=left]
(base) PS C:\> python
>>> 0.1
0.1
>>> 0.1 + 7
7.1
>>> (0.1 + 7) - 7
0.09999999999999964
\end{Verbatim}
Ilzītei radās aizdomas, ka pēdējais rezultāts radies noapaļojot, 
tāpēc viņa izteica skaitli $0.1 = \frac{1}{10}$ kā bezgalīgu summu:
$$0.1 = \frac{d_1}{2} + \frac{d_2}{2^2} + \frac{d_3}{2^3} + \frac{d_4}{2^4} + \ldots,$$
kur ikviens $d_i$ ir {\em binārā pieraksta cipars}, kas ir vai nu $0$, vai $1$. 

Atrast periodisku virkni ar binārajiem cipariem, kas precīzi izsaka $0.1$. Un atrast, 
līdz kurai pozīcijai skaitlis varētu būt bijis noapaļots, ka tika iegūts rezultāts {\tt 0.09999999999999964}. 
\end{problem}




%Farey sequence
\vspace{10pt}
\begin{problem}
Zināms, ka racionāla daļa $\frac{p}{q}$ apmierina nevienādības:
$$\frac{2019}{2020} < \frac{p}{q} < \frac{2020}{2021}.$$
Starp visām šādām daļām atrast to, kurai saucējs $q$ ir vismazākais.
\end{problem}


%Farey sequence
\vspace{10pt}
\begin{problem}
\begin{enumerate}[(A)]
\item Vai skaitlis $\sqrt[3]{2}$ ir racionāls vai iracionāls?
\item Pierādīt vai apgāzt sekojošu apgalvojumu:\\
``Katram reālam skaitlim $x$, ja $x^2$ ir iracionāls, tad arī $x^3$ ir iracionāls.''
\end{enumerate}
\end{problem}



\end{document}