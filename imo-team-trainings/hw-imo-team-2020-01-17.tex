\documentclass[11pt]{article}
\usepackage{ucs}
\usepackage[utf8x]{inputenc}
\usepackage{changepage}
\usepackage{graphicx}
\usepackage{amsmath}
\usepackage{gensymb}
\usepackage{amssymb}
\usepackage{enumerate}
\usepackage{tabularx}
\usepackage{lipsum}
\usepackage{amsthm}
\usepackage{thmtools}


\usepackage{fontspec} % loaded by polyglossia, but included here for transparency 
\usepackage{polyglossia}

\usepackage{xeCJK}
\setCJKmainfont{SimSun}
\setmainlanguage{russian} 
\setotherlanguage{english}

\newfontfamily\cyrillicfont[Script=Cyrillic]{Times New Roman}
\newfontfamily\cyrillicfontsf[Script=Cyrillic]{Arial}
\newfontfamily\cyrillicfonttt[Script=Cyrillic]{Courier New}

\oddsidemargin 0.0in
\evensidemargin 0.0in
\textwidth 6.27in
\headheight 1.0in
\topmargin 0.0in
\headheight 0.0in
\headsep 0.0in
%\textheight 9.69in
\textheight 9.00in
 
\setlength\parindent{0pt}

\newenvironment{myenv}{\begin{adjustwidth}{0.4in}{0.4in}}{\end{adjustwidth}}
\renewcommand{\abstractname}{Anotācija}
\renewcommand\refname{Atsauces}

%\newenvironment{uzdevums}[1][\unskip]{%
%\vspace{3mm}
%\noindent
%\textbf{#1:}
%\noindent}
%{}

% (4;10;12;17)
% (p1.19;5;15;20)

% http://tex.stackexchange.com/questions/196961/thmtools-declaration-for-theorem-and-proof
\declaretheoremstyle[headfont=\normalfont\bfseries,notefont=\mdseries\bfseries,bodyfont = \normalfont,headpunct={:}]{normalhead}
\declaretheorem[name={Uzdevums}, style=normalhead,numberwithin=section]{problem}

%\def\changemargin#1#2{\list{}{\rightmargin#2\leftmargin#1}\item[]}
\def\changemargin#1#2{\list{}\item[]}
\let\endchangemargin=\endlist 


\newcommand{\subf}[2]{%
  {\small\begin{tabular}[t]{@{}c@{}}
  #1\\#2
  \end{tabular}}%
}



\newcounter{alphnum}
\newenvironment{alphlist}{\begin{list}{(\Alph{alphnum})}{\usecounter{alphnum}\setlength{\leftmargin}{2.5em}} \rm}{\end{list}}

\newenvironment{zhtext}{\fontfamily{MS PGothic}\selectfont}{\par}


\makeatletter
\let\saved@bibitem\@bibitem
\makeatother

\usepackage{bibentry}
%\usepackage{hyperref}

\newenvironment{tulkojums}[1][\unskip]{%
\begin{changemargin}{8mm}{8mm}
\fontsize{9}{11}
\selectfont
\textbf{#1:}
}
{ 
\fontsize{12}{14}
\selectfont
\end{changemargin}
}

\setcounter{section}{1}


\begin{document}

\begin{center}
{\Large \bf Uzdevumi 2020.g. 17.\ janvāra nodarbībai}
\end{center}

\vspace{10pt}

\begin{problem}
Prove that there exist infinitely many positive integers $n$ 
such that the largest prime divisor
of $n^4 + n^2 + 1$ is equal to the largest 
prime divisor of $(n+1)^4 + (n+1)^2 + 1$.
\end{problem}

\begin{problem}
Fix an integer $k \geq 2$. Two players, called Ana and Banana, 
play the following {\em game of
numbers}: Initially, some integer $n \geq k$ gets written 
on the blackboard. Then they take moves
in turn, with Ana beginning. A player making a move 
erases the number $m$ just written on the
blackboard and replaces it by some number $m'$ with 
$k \leq m' < m$ that is coprime to $m$. The first
player who cannot move anymore loses.

An integer $n \geq k$ is called good if Banana has a winning 
strategy when the initial number is $n$, and bad otherwise.

Consider two integers $n, n' \geq k$ with the property 
that each prime number $p \leq k$ divides $n$ if
and only if it divides $n'$.
Prove that either both $n$ and $n'$
are good or both are bad.
\end{problem}

\begin{problem}
Let $n > 1$ be a given integer. Prove that infinitely 
many terms of the sequence $\left( a_k \right)_{k \geq 1}$,
defined by
$$a_k = \left\lfloor \frac{n^k}{k} \right\rfloor,$$
are odd. (For a real number $x$, $\lfloor x \rfloor$ denotes 
the largest integer not exceeding $x$.)
\end{problem}

\begin{problem}
Find all triples $(p,x,y)$ consisting of a prime number $p$ 
and two positive integers $x$ and $y$
such that $x^{p-1} + y$  and $x + y^{p-1}$ are both powers of $p$.
\end{problem}


\end{document}


